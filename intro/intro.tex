\chapter{Proteomic Acquisition Strategies for Mass Spectrometry}

\section{Proteomics and Mass Spectrometry}

\subsection{Proteomics.}
Proteomics is the large-scale study of proteins. Proteins are an essential part of life and every living thing contains proteins. If genes are the blueprints for life, then proteins are the construction workers, building supplies, and tools that empower and sustain life. Their involvement in life ranges from the diseases and aliments that cause impairment to the therapeutics and medicines that cure them. From agriculture and food that provides energy to motion and structure that defines the shape of cells. There is very little in life that proteins do not affected. Understanding their role and function in biological systems is an important goal of life sciences. 

When compared to deoxyribonucleic acid (DNA), proteins are very similar. Both are long, linear chemical polymers comprised of different monomers: DNA has 4 nucleotide bases (G,T,A,C) and proteins are made up of roughly 20 amino acids (A,C,D,E,F,G,H,I,K,L,M,N,P,Q,R,S,T,V,W,Y). The sequential order of these monomers in both DNA and proteins encodes information. The information stored in DNA is primarily used to construct proteins and can be thought of as an instruction book, or blueprint. On the other hand, protein sequences isn't used to store instructions, but rather their structure. Proteins fold into complex three dimensional structures depending on their amino acid sequence and it is these 3D structures that provide the different mechanical and chemical functions for life to work.

A single organism will often contain hundreds or thousands of different protein sequences---depending on their complexity, and that set of proteins is called the proteome. For example, the proteome of baker's yeast (\emph{Saccharomyces cerevisiae}) contains \textasciitilde4,600 different protein sequences. The human (\emph{Homo sapiens}) proteome is much larger, with nearly 12,000 different protein sequences known to be expressed. Often in biology it is important to know how an organism's proteins change when exposed to different conditions. To detect thousands of proteins is a daunting task, and it has only been in the last two decades have technologies been capable of identifying large portions of proteomes. 

The main technology used is mass spectrometry (MS)

\subsection*{Mass Spectrometry.}
Mass spectrometry is a powerful analytical tool. A mass spectrometer is an instrument that measures the mass of ions and is comprised of three parts: 1) an ionization source, 2) a mass analyzer, and 3) a detector. 

Gas phase ions (negative or positive) are first generated from an analyte in either the solid or liquid phases by the ionization source. There are many ionization techniques in use today, hard-ionization methods such as electron impact ionization (EI) causes the molecule to fragment during ionization. Softer methods that minimize fragmentation include fast-atom bombardment (FAB), matrix assisted laser desorption ionization (MALDI), electrospray ionization (ESI), and chemical ionization (CI), among others. ESI has become the primary method for large-scale proteomic studies because of the ease it which it can couple to a front-end separation methods such as liquid chromatography (LC). Separation is needed in most proteomic experiments because the samples consists of thousands of peptides, and separating them prior to ioniziation is an important factor in increasing sensitivity. 

The second part of the mass spectrometer is the mass analyzer, which separates the gas phase ions based on their mass-to-charge ratios (\mz{}). There are many different types of mass analyzers: magnetic sector, time-of-flight (TOF), quadrupole mass filters, ion-traps, orbitraps, and fourier transform ion cyclotron resonsnace (FT-ICR). Each rely on 


Since every chemical compound has a mass and can potentially be ionized, they can be mass analyzed with a mass spectrometer.

Determining the mass of analytes is useful in and of itself, but not so much when proteins are mass analyzed. This is because the sequence of amino acids is hidden, i.e., there are many different orderings of amino acids that would give you the same total mass. Fortunately, mass spectrometers are multifaceted instruments, and provide ways for determining amino acid sequence.


%history
%DDA/DIA
%Dynamic Exclusion
%Decision Tree
%inseq
%eoa
%future
