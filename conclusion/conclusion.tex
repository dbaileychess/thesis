\chapter{The Future of Intelligent Data Acquisition Methods}

The work presented here is on the forefront of mass spectrometry based acquisition methods. 

There are two main methods for increasing the intelligence of mass spectrometers. The first would be if there is a home built mass spectrometer that is controlled by custom controllers and software. The other way is to modify and extend commercially available mass spectrometers with the desired abilities. For large scale proteomic work, the former approach is not straightforward, as a vast majority of large-scale publication use commercial instruments for data acquisition. Custom built mass spectrometers often focus on a very specific task (e.g., new mass analyzer, new dissociation technique, etc.) and rarely are geared for high-performance, large-scale protein experiments. Commercial instruments, on the other hand, are primarily developed to take the best technologies available and combine them into one unified package. This results in a powerful and stable instrument that can handle the largest experiments. However, in order to protect their intellectual property, instrument vendors are usually highly restrictive in how their instruments are used. This makes implementing novel acquisition methods very difficult, and therefore general acceptance of new methods is slow. Thus, the most important factor in the future of intelligent data acquisition methods is increasing the accessibility and availability.

\section{Accessibility of Intelligent Mass Spectrometers}
Probably the best way to propel the development of intelligent acquisition methods forward is to increase its accessibility and availability to researchers. This is challenging since instrument vendors are highly protective of their products; they have to protect their intellectual property and public imagine while providing state-of-the-art technologies to consumers. They cannot release access to their control logic in fears of competitors gaining an edge. They also worry that supporting third-party programs for their instruments could damage their reputation. Our lab, which has developed multiple technologies now commercialized, know first hand the care instrument vendors take in releasing third-party technologies to the general consumer. 




\section*{Need}
Behind any feature or tool, there always lies the question ''Is this needed?'' Can some other method provide benefits 


\section*{title}
